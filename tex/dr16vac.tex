% e.g., http://adsabs.harvard.edu/abs/2014ApJ...790..127B

\documentclass[modern]{aastex62}
\usepackage{amsmath}

% Load common packages
% \usepackage{microtype}  % ALWAYS!
\usepackage{amsmath}
\usepackage{amsfonts}
\usepackage{amssymb}
\usepackage{booktabs}

\usepackage{graphicx}
\usepackage{color}

% \definecolor{cbblue}{HTML}{3182bd}
% \usepackage{hyperref}
% \definecolor{linkcolor}{rgb}{0.02,0.35,0.55}
% \definecolor{citecolor}{rgb}{0.45,0.45,0.45}
% \hypersetup{colorlinks=true,linkcolor=linkcolor,citecolor=citecolor,
%             filecolor=linkcolor,urlcolor=linkcolor}
% \hypersetup{pageanchor=true}

\newcommand{\documentname}{\textsl{Article}}
\newcommand{\sectionname}{Section}
\renewcommand{\figurename}{Figure}
\newcommand{\equationname}{Equation}
\renewcommand{\tablename}{Table}

% Missions
\newcommand{\project}[1]{\textsl{#1}}

% Packages / projects / programming
\newcommand{\package}[1]{\textsl{#1}}
\newcommand{\acronym}[1]{{\small{#1}}}
\newcommand{\github}{\package{GitHub}}
\newcommand{\python}{\package{Python}}
\newcommand{\emcee}{\project{emcee}}

% For referee
\newcommand{\changes}[1]{{\color{red} #1}}

% Stats / probability
\newcommand{\given}{\,|\,}
\newcommand{\norm}{\mathcal{N}}

% Maths
\newcommand{\dd}{\mathrm{d}}
\newcommand{\transpose}[1]{{#1}^{\mathsf{T}}}
\newcommand{\inverse}[1]{{#1}^{-1}}
\newcommand{\argmin}{\operatornamewithlimits{argmin}}
\newcommand{\mean}[1]{\left< #1 \right>}

% Unit shortcuts
\newcommand{\msun}{\ensuremath{\mathrm{M}_\odot}}
\newcommand{\kms}{\ensuremath{\mathrm{km}~\mathrm{s}^{-1}}}
\newcommand{\mps}{\ensuremath{\mathrm{m}~\mathrm{s}^{-1}}}
\newcommand{\pc}{\ensuremath{\mathrm{pc}}}
\newcommand{\kpc}{\ensuremath{\mathrm{kpc}}}
\newcommand{\kmskpc}{\ensuremath{\mathrm{km}~\mathrm{s}^{-1}~\mathrm{kpc}^{-1}}}

% Misc.
\newcommand{\bs}[1]{\boldsymbol{#1}}

% Astronomy
\newcommand{\DM}{{\rm DM}}
\newcommand{\feh}{\ensuremath{{[{\rm Fe}/{\rm H}]}}}
\newcommand{\df}{\acronym{DF}}
\newcommand{\logg}{\ensuremath{\log g}}
\newcommand{\Teff}{\ensuremath{T_{\textrm{eff}}}}

% TO DO
\newcommand{\todo}[1]{{\color{red} TODO: #1}}

\newcommand{\DR}{\acronym{DR}16}
\newcommand{\apogee}{\project{\acronym{APOGEE}}}
\newcommand{\sdssiv}{\project{\acronym{SDSS-IV}}}
\newcommand{\thejoker}{\project{The~Joker}}


\shorttitle{Stuff}
\shortauthors{Price-Whelan et al.}

\begin{document}

\title{Stellar and substellar companions in APOGEE DR16: \\
       Catalog of XX systems}

\author[0000-0003-0872-7098]{Adrian~M.~Price-Whelan}
\affiliation{Center for Computational Astrophysics, Flatiron Institute,
             Simons Foundation, 162 Fifth Avenue, New York, NY 10010, USA}
\email{aprice-whelan@flatironinstitute.org}
\correspondingauthor{Adrian M. Price-Whelan}

% \author[0000-0003-2866-9403]{David~W.~Hogg}
% \affiliation{Max-Planck-Institut f\"ur Astronomie,
%              K\"onigstuhl 17, D-69117 Heidelberg, Germany}
% \affiliation{Center for Cosmology and Particle Physics,
%              Department of Physics,
%              New York University, 726 Broadway,
%              New York, NY 10003, USA}
% \affiliation{Center for Data Science,
%              New York University, 60 Fifth Ave,
%              New York, NY 10011, USA}
% \affiliation{Flatiron Institute,
%              Simons Foundation,
%              162 Fifth Avenue,
%              New York, NY 10010, USA}

% \author[0000-0003-4996-9069]{Hans-Walter~Rix}
% \affiliation{Max-Planck-Institut f\"ur Astronomie,
%              K\"onigstuhl 17, D-69117 Heidelberg, Germany}

\author{Others}

\begin{abstract}
% Context
Binary star systems provide key context and constraints for nearly all subfields in astrophysics, yet precise inferences about binary star populations typically rely on just hundreds of stars nearest to the sun.
Detailed knowledge of binary star population properties is therefore
% Aims
Contemporary stellar surveys already observe large numbers of stars
% Methods
We have a big hammer
% Results
We produce posterior samplings assuming...SB1 and two-body systems...
We also produce a catalog of binary systems...we select on evidence or parameter estimation...
% Conclusions
Binaries all over HR diagram...but we find ??
We find interesting low-mass and high-mass systems...
\end{abstract}

\keywords{}


\section{Introduction} \label{sec:intro}

A better understanding the population of binary stars throughout our galaxy is important for all aspects of astrophysics. For example, ... stellar-mass binary black hole mergers to stellar populations in high-redshift galaxies \citep[e.g.,][]{Breivik:2019, Rix:2019}.

We need to understand the population statistics of stellar multiplicity and their variations with stellar type, chemistry, and dynamical environment.

This problem spans a wide range of timescales, wavelengths, and many current and near-future stellar surveys (Gaia, APOGEE, LAMOST, SDSS-V, etc.) have the capacity to deliver samples of binary stars and stellar companions orders of magnitude larger than are presently known, throughout all stages of stellar evolution.
The dream: combine all if this and learn ...
This has only been comprehensively done for a sample of a few hundred stars in solar neighborhood.

As a step towards large-scale population inference, we focus here on multi-epoch spectroscopic data from the APOGEE surveys.
The challenge: data are often sparse, most individual systems are not determined uniquely.
We have machinery to generate full samplings over orbital parameters for radial velocity data, under the assumptions listed in The Joker paper (SB1, etc.).
We run this sampler on all of APOGEE DR16 with some quality cuts and release the resulting catalog of posterior samples, along with some summary metadata for unimodal / uniquely determined orbits.

We use these samples to infer the companion period-eccentricity distribution over the HR diagram as an initial demonstration of hierarchical inference.

\section{Data} \label{sec:data}

We primarily use spectroscopic data from data release 16 (\DR) of the \apogee\ surveys (\citealt{Majewski:2017, Abolfathi:2017, XXXX:2019}).
\apogee\ is a component of the Sloan Digital Sky Survey IV (\sdssiv; \citealt{Gunn:2006, Blanton:2017}) and the main survey is designed to map stars across much of the Milky Way disk by obtaining high-resolution ($R \sim 22,500$) infrared ($H$-band) spectroscopy of stars selected in a homogenous way.

Targets are selected with simple color and brightness cuts, but the survey uses
fiber-plugged plates with a maximum of 300 fibers per each $\approx
1.5~\textrm{deg}^2$ field of view, leading to ``pencil-beam''-like sampling of
the Galactic stellar distribution.
In order to meet signal-to-noise ratio requirements, most \apogee\ stars are
observed multiple times in a series of ``visits,'' typically with at least one
visit separated by a month or more in order to help identify binary stars.



Data from APOGEE surveys, DR16 - briefly summarize improvements over DR14 that are directly relevant to RVs (main sequence stars). Caveat: SB1 will be bad assumption for many MS systems.

Figure: DR16 logg-Teff diagram for all star we run on

Figure: N visit plot like in previous paper, some quantification of baseline

\subsection{Visit velocity error calibration} \label{sec:visitcalib}

Visit velocity errors underestimated, so we perform our own calibration.

Describe exoplanet survey crossmatch and calibration bullshit.

This ends up being conservative: we effectively force low SNR visit to a SNR-dependent threshold rv err, high SNR stars get adjusted too with a global floor to RV error.

Figure: raw visit errors vs. SNR and then adjusted visit errors


\section{Methods} \label{sec:methods}

Describe The Joker.

The bug: the fix. The new prior on K.

Describe robust constant fit.

\section{A catalog of binary stars} \label{sec:catalog}

Any catalog will have some arbitrary cuts. Here we cut on likelihood ratio of binary star model vs. constant model (not fully marginal likelihoods, so cut won't be at 0!).

We find XX stars with companions - show simple plots over HR diagram for cuteness.

Gold sample: phase coverage statistics?

TODO: what about bimodal samplings, or samplings with small range in ln-period?

Discuss RGB vs. MS primaries differently

Figure: Some cute examples

Figure: Fraction of stars with companions over HR diagram

\section{Inferring shit with a hierarchical model} \label{sec:Pe}

\begin{align}
    z &= \ln\,P\\
    \bs{\theta} &= (z, e)\\
    \bs{\alpha} &= (k, z_0, \alpha_0)\\
    p(\bs{\theta} \given \bs{\alpha}) &= p(z) \, p(e \given z)\\
    p(z) &= \norm\left(z \given \mu_z, \sigma_z^2 \right)\\
    p(e \given z) &= \alpha(z) \, p_1(e) + (1-\alpha(z)) \, p_2(e)\\
    f(z\,;\,k,z_0) &= \frac{1}{1 + e^{-k\,(z - z_0)}}\\
    \alpha(z \,;\, k, z_0, \alpha_0) &= (1-\alpha_0) \, f(z \,;\, k, z_0) + \alpha_0\\
    p_1(e) &= B(e \given a_1, b_1)\\
    p_2(e) &= B(e \given a_2, b_2)
\end{align}

\begin{align}
    p(D \given \bs{\alpha}) &= \int \dd\bs{\theta} \,
        p(D \given \bs{\theta}) \, p(\bs{\theta} \given \bs{\alpha})\\
    &\approx \frac{\mathcal{Z}}{K} \sum_k^K
        \frac{p(\bs{\theta}_k \given \bs{\alpha})}
            {p(\bs{\theta}_k \given \bs{\alpha}_0)}\\
    \bs{\theta}_k &\sim p(\bs{\theta} \given D, \bs{\alpha}_0)
\end{align}

- Period, eccentricity distributions
- Mass-ratio distribution
     - Use C,N abundances for RGB and logg, teff, fe/h for main sequence
     - Calibrate against asteroseismology / APOKASC


Lars: "Any idea of the incidence of stars which did an engulfment on their way UP the RGB from the Luminosity of the clump to that of the RGB TIP? "

% Note: some or all of these may be broken out into separate papers!

% \subsection{Companions along the RGB}
% Compare RGB vs. RC - see engulfment?
%
% \subsection{Brown dwarfs}
% So many!
%
% \subsection{Binary population statistics}
% Simple hierarchical inference?
%
% \subsection{TRGB asteroseismology}
%
% \subsection{Binary star populations in open clusters}
%
% \subsection{Binary star populations in globular clusters}

% Ack: Lars Bildsten


\software{
    Astropy \citep{astropy, astropy:2018},
    emcee \citep{emcee},
    gala \citep{gala},
    IPython \citep{ipython}
}

\bibliographystyle{aasjournal}
\bibliography{dr16vac}

\end{document}
