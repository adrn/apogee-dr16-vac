% e.g., http://adsabs.harvard.edu/abs/2014ApJ...790..127B

\documentclass[modern]{aastex62}
\usepackage{amsmath}

% Load common packages
% \usepackage{microtype}  % ALWAYS!
\usepackage{amsmath}
\usepackage{amsfonts}
\usepackage{amssymb}
\usepackage{booktabs}

\usepackage{graphicx}
\usepackage{color}

% \definecolor{cbblue}{HTML}{3182bd}
% \usepackage{hyperref}
% \definecolor{linkcolor}{rgb}{0.02,0.35,0.55}
% \definecolor{citecolor}{rgb}{0.45,0.45,0.45}
% \hypersetup{colorlinks=true,linkcolor=linkcolor,citecolor=citecolor,
%             filecolor=linkcolor,urlcolor=linkcolor}
% \hypersetup{pageanchor=true}

\newcommand{\documentname}{\textsl{Article}}
\newcommand{\sectionname}{Section}
\renewcommand{\figurename}{Figure}
\newcommand{\equationname}{Equation}
\renewcommand{\tablename}{Table}

% Missions
\newcommand{\project}[1]{\textsl{#1}}

% Packages / projects / programming
\newcommand{\package}[1]{\textsl{#1}}
\newcommand{\acronym}[1]{{\small{#1}}}
\newcommand{\github}{\package{GitHub}}
\newcommand{\python}{\package{Python}}
\newcommand{\emcee}{\project{emcee}}
\newcommand{\thejoker}{\project{The Joker}}

% For referee
\newcommand{\changes}[1]{{\color{red} #1}}

% Stats / probability
\newcommand{\given}{\,|\,}
\newcommand{\norm}{\mathcal{N}}

% Maths
\newcommand{\dd}{\mathrm{d}}
\newcommand{\transpose}[1]{{#1}^{\mathsf{T}}}
\newcommand{\inverse}[1]{{#1}^{-1}}
\newcommand{\argmin}{\operatornamewithlimits{argmin}}
\newcommand{\mean}[1]{\left< #1 \right>}

% Unit shortcuts
\newcommand{\msun}{\ensuremath{\mathrm{M}_\odot}}
\newcommand{\kms}{\ensuremath{\mathrm{km}~\mathrm{s}^{-1}}}
\newcommand{\mps}{\ensuremath{\mathrm{m}~\mathrm{s}^{-1}}}
\newcommand{\pc}{\ensuremath{\mathrm{pc}}}
\newcommand{\kpc}{\ensuremath{\mathrm{kpc}}}
\newcommand{\kmskpc}{\ensuremath{\mathrm{km}~\mathrm{s}^{-1}~\mathrm{kpc}^{-1}}}

% Misc.
\newcommand{\bs}[1]{\boldsymbol{#1}}

% Astronomy
\newcommand{\DM}{{\rm DM}}
\newcommand{\feh}{\ensuremath{{[{\rm Fe}/{\rm H}]}}}
\newcommand{\df}{\acronym{DF}}

% TO DO
\newcommand{\todo}[1]{{\color{red} TODO: #1}}


\shorttitle{Stuff}
\shortauthors{Price-Whelan et al.}

\begin{document}

\title{A large catalog of stellar and substellar companions from the APOGEE surveys}

\author[0000-0003-0872-7098]{Adrian~M.~Price-Whelan}
\affiliation{TODO: Flatiron}
\affiliation{Department of Astrophysical Sciences,
             Princeton University, Princeton, NJ 08544, USA}
\email{aprice-whelan@flatironinstitute.org}
\correspondingauthor{Adrian M. Price-Whelan}


\begin{abstract}
Stuff!
\end{abstract}

\keywords{}


\section{Introduction} \label{sec:intro}

Big surveys (APOGEE, Gaia, LAMOST, etc.) have the capacity to deliver huge samples of binary stars, stellar companions - much larger than samples upon which binary star evolution is pinned.

But a challenge is sparsity of data: traditional methods cannot easily use most data to do population studies if orbital parameters of each source not determined uniquely.

We have machinery to generate full samplings over orbital parameters for radial velocity data, under the assumptions listed in The Joker paper (SB1, etc.).

We run this sampler on all of APOGEE DR16 with some quality cuts.

We release the resulting catalog of posterior samples, along with some summary metadata for unimodal / uniquely determined orbts.

Here we also use these samples to infer the companion period-eccentricity distribution over the HR diagram.

\section{Data} \label{sec:data}

Data from APOGEE surveys, DR16 - briefly summarize improvements over DR14 that are directly relevant to RVs (main sequence stars). Caveat: SB1 will be bad assumption for many MS systems.

Figure: DR16 logg-Teff diagram for all star we run on
Figure: N visit plot like in previous paper

\subsection{Visit velocity error calibration} \label{sec:visitcalib}

Visit velocity errors underestimated, so we perform our own calibration.

Describe exoplanet survey crossmatch and calibration bullshit.

This ends up being conservative: we effectively force low SNR visit to a SNR-dependent threshold rv err, high SNR stars get adjusted too with a global floor to RV error.

Figure: raw visit errors vs. SNR and then adjusted visit errors


\section{Methods} \label{sec:methods}

Describe The Joker.

Describe robust constant fit.

\section{A catalog of binary stars} \label{sec:catalog}

Any catalog will have some arbitrary cuts. Here we cut on likelihood ratio of binary star model vs. constant model (not fully marginal likelihoods, so cut won't be at 0!).

We find XX stars with companions - show simple plots over HR diagram for cuteness.

Gold sample: phase coverage statistics?

TODO: what about bimodal samplings, or samplings with small range in ln-period?

Figure: Some cute examples

Figure: Fraction of stars with companions over HR diagram

\section{Inferring shit with a hierarchical model} \label{sec:Pe}

\begin{align}
    z &= \ln\,P\\
    \bs{\theta} &= (z, e)\\
    \bs{\alpha} &= (k, z_0, a, b)\\
    p(\bs{\theta} \given \bs{\alpha}) &= p(z) \, p(e \given z)\\
    p(z) &= \norm\left(z \given \mu_z, \sigma_z^2 \right)\\
    p(e \given z) &= \alpha(z) \, p_1(e) + (1-\alpha(z)) \, p_2(e)\\
    f(z\,;\,k,z_0) &= \frac{1}{1 + e^{-k\,(z - z_0)}}\\
    \alpha(z \,;\, k, z_0, a, b) &= a \, f(z \,;\, k, z_0) + b\\
    p_1(e) &= B(e \given \text{fixed})\\
    p_2(e) &= B(e \given \text{fixed})
\end{align}

\begin{align}
    p(D \given \bs{\alpha}) &= \int \dd\bs{\theta} \,
        p(D \given \bs{\theta}) \, p(\bs{\theta} \given \bs{\alpha})\\
    &\approx \frac{\mathcal{Z}}{K} \sum_k^K
        \frac{p(\bs{\theta}_k \given \bs{\alpha})}
            {p(\bs{\theta}_k \given \bs{\alpha}_0)}\\
    \bs{\theta}_k &\sim p(\bs{\theta} \given D, \bs{\alpha}_0)
\end{align}

% Note: some or all of these may be broken out into separate papers!

% \subsection{Companions along the RGB}
% Compare RGB vs. RC - see engulfment?
%
% \subsection{Brown dwarfs}
% So many!
%
% \subsection{Binary population statistics}
% Simple hierarchical inference?
%
% \subsection{TRGB asteroseismology}
%
% \subsection{Binary star populations in open clusters}
%
% \subsection{Binary star populations in globular clusters}


\software{
    Astropy \citep{astropy, astropy:2018},
    emcee \citep{emcee},
    gala \citep{gala},
    IPython \citep{ipython}
}

\bibliographystyle{aasjournal}
\bibliography{dr16vac}

\end{document}
